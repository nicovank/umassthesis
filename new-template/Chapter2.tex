\chapter{ROLE OF GENE ABC IN PATHOGEN TRANSMISSION}

\section{WXYZ Virus}

Gene ABC regulates critical feeding behavior during the larval stage. This regulation is critical for the proper development of the mosquito. We hypothesize that disruption of this regulation could have long-lasting impact on the ability of the mosquito to transmit pathogens, namely WXYZ virus (see Fig. \ref{fig:virus}). WXYZ Virus is a highly virulent pathogen common throughout much of the world \citep{chen2023}. 
\begin{figure}[h!]
    \centering
    \includegraphics[width=0.5\linewidth]{Images/virus.png}
    \caption{Computer Model of the Structure of WXYZ Virus}
    \label{fig:virus}
\end{figure}
We used gene-editing technology to disrupt the expression of the gene in the larval stage. We then exposed control and experimental adults to the virus and measured the rate of infection (Table \ref{infection}). We performed five different replicate trials of this experiment to ensure any variation we observed was consistent. We observed significantly higher rates of infection in the experimental group of mosquitoes. This difference was observed in each of the five experimental replicates we conducted. To confirm the significance of this difference, we performed appropriate statistical analysis. This consistency across replicates suggests that larval expression of Gene ABC does in fact play an important role in the long-term immunity of mosquitoes. This data may provide some insight into the role developmental genetics plays in influencing the subsequent transmission of pathogens by medically-relevant insect species.

\begin{table}[t!]
\centering
\begin{tabular}{crr}
\toprule
Trial & Control \% &  Experimental \% \\
\midrule
A & 33 & 68 \\
B & 42 & 82 \\
C & 54 & 90 \\
D & 55 & 93 \\
E & 46 & 80 \\
\bottomrule
\end{tabular}
\caption{\centering Infection Prevalence in Control and Experimental Mosquitoes}
\label{infection}
\end{table}
